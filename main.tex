\documentclass[11pt, a4paper, leqno]{article}
\usepackage{a4wide}
\usepackage[T1]{fontenc}
\usepackage[utf8]{inputenc}
\usepackage{float, afterpage, rotating, graphicx}
\usepackage{epstopdf}
\usepackage{longtable, booktabs, tabularx}
\usepackage{fancyvrb, moreverb, relsize}
\usepackage{eurosym, calc}
\usepackage{amsmath, amssymb, amsfonts, amsthm, bm}
\usepackage{caption}
\usepackage{mdwlist}
\usepackage{xfrac}
\usepackage{setspace}
\usepackage[dvipsnames]{xcolor}
\usepackage{subcaption}
\usepackage{minibox}
\usepackage[
    natbib=true,          % Usa citazioni in stile author-year
    bibencoding=inputenc,
    bibstyle=authoryear,  % Stile bibliografico autore-anno
    citestyle=authoryear,
    maxcitenames=3,       % Mostra massimo 3 autori prima di usare "et al."
    maxbibnames=10,       % Mostra fino a 10 autori in bibliografia
    useprefix=false,
    sorting=nyt,          % Ordina per autore, anno, titolo
    giveninits=true,      % Usa le iniziali per i nomi propri
    backend=biber         % Usa Biber invece di BibTeX
]{biblatex}


\AtBeginDocument{\toggletrue{blx@useprefix}}
\AtBeginBibliography{\togglefalse{blx@useprefix}}
\setlength{\bibitemsep}{1.5ex}
\addbibresource{DOT.bib}

\usepackage[unicode=true]{hyperref}
\hypersetup{
    colorlinks=true,
    linkcolor=black,
    anchorcolor=black,
    citecolor=NavyBlue,
    filecolor=black,
    menucolor=black,
    runcolor=black,
    urlcolor=NavyBlue
}
\renewbibmacro*{author}{%
    \printnames{author}%
    \setunit{\addcomma\space}%
}
\renewcommand*{\newunitpunct}{\addcomma\space}
\renewbibmacro*{in:}{}
\DeclareBibliographyDriver{article}{%
    \usebibmacro{author}%
    \usebibmacro{title}%
    \newunit
    \printfield{journaltitle}%
    \setunit{\addcomma\space}%
    \printfield{year}%
    \setunit{\addcomma\space}%
    \printfield{volume}%
    \setunit{\addcomma\space}%
    \printfield{number}%
    \setunit{\addcomma\space}%
    \printfield{pages}%
    \newunit
}
\widowpenalty=10000
\clubpenalty=10000

\setlength{\parskip}{1ex}
\setlength{\parindent}{0ex}
\setstretch{1.5}

\title{Mapping Occupational Traits from the 1965 Third Edition of the Dictionary of Occupational Titles to the European ISCO 08 Classification \thanks{This document serves as an auxiliary source of information on the data used for the analysis in the paper \textit{``Working Conditions and Health at Work Over the Lifecycle: Evidence from Europe"}.}}

\author{Belloni, Lucifora, Micera} 

\date{}


\begin{document}

\maketitle
\vspace{-3em}

\begin{center}  \href{https://github.com/paolomicera/DOT3rd-ISCO08/blob/main/main.pdf}{\textbf{Most Updated Version}}
\end{center}
\section{Introduction}
This document provides detailed information on the collection of data on working conditions from the digitalized version of the third edition of the Dictionary of Occupational Titles (DOT) and its conversion to the European classification system ISCO-08. 
The Dictionary of Occupational Titles (DOT) was first published in 1939 and underwent four major updates: in 1949, 1965, 1977, and its final revision in 1991. Since then, it has been replaced by the Occupational Information Network (O*NET). The DOT is a comprehensive classification system of occupational information, which includes a wide range of variables, such as job titles, job descriptions, and occupational characteristics. The DOT was developed by the U.S. Department of Labor and was used for various purposes, such as job placement, vocational guidance, and labor market analysis. The contribution of this document is twofold. First, it provides a detailed description of the data collection process from the digitalized version of the 1965 third edition of the DOT. Second, it describes the mapping of the occupational traits from the DOT to the European classification system ISCO-08.








\newpage
\nocite{*}
\printbibliography

\end{document}
