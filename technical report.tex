\documentclass[11pt, a4paper, leqno]{article}
\usepackage{a4wide}
\usepackage[T1]{fontenc}
\usepackage[utf8]{inputenc}
\usepackage{float, afterpage, rotating, graphicx}
\usepackage{epstopdf}
\usepackage{longtable, booktabs, tabularx}
\usepackage{fancyvrb, moreverb, relsize}
\usepackage{eurosym, calc}
\usepackage{amsmath, amssymb, amsfonts, amsthm, bm}
\usepackage{caption}
\usepackage{mdwlist}
\usepackage{xfrac}
\usepackage[dvipsnames]{xcolor}
\usepackage{subcaption}
\usepackage{minibox}
\usepackage[bottom]{footmisc} 
\setlength{\skip\footins}{12pt} % Aumenta lo spazio tra testo e footnote (prova a modificare 12pt)
\usepackage[
    natbib=true,
    bibencoding=inputenc,
    bibstyle=authoryear,  % <--- MAKE SURE THIS MATCHES YOUR STYLE
    citestyle=authoryear,
    maxcitenames=3,
    maxbibnames=10,
    useprefix=false,
    sortcites=true,
    backend=biber
]{biblatex}
\addbibresource{refs.bib}  % Make sure this points to the correct .bib file
\usepackage[left=0.8in, right=0.8in, top=1in, bottom=1in]{geometry}
\linespread{0.7}  % Reduces the spacing



\AtBeginDocument{\toggletrue{blx@useprefix}}
\AtBeginBibliography{\togglefalse{blx@useprefix}}
\setlength{\bibitemsep}{1.5ex}
\addbibresource{DOT.bib}

\usepackage[unicode=true]{hyperref}
\hypersetup{
    colorlinks=true,
    linkcolor=black,
    anchorcolor=black,
    citecolor=NavyBlue,
    filecolor=black,
    menucolor=black,
    runcolor=black,
    urlcolor=NavyBlue
}
\renewbibmacro*{author}{%
    \printnames{author}%
    \setunit{\addcomma\space}%
}
\renewcommand*{\newunitpunct}{\addcomma\space}
\renewbibmacro*{in:}{}
\DeclareBibliographyDriver{article}{%
    \usebibmacro{author}%
    \usebibmacro{title}%
    \newunit
    \printfield{journaltitle}%
    \setunit{\addcomma\space}%
    \printfield{year}%
    \setunit{\addcomma\space}%
    \printfield{volume}%
    \setunit{\addcomma\space}%
    \printfield{number}%
    \setunit{\addcomma\space}%
    \printfield{pages}%
    \newunit
}
\widowpenalty=10000
\clubpenalty=10000

\setlength{\parskip}{1ex}
\setlength{\parindent}{0ex}
\usepackage{setspace}
\setstretch{1.2}
\usepackage{titlesec}
\titleformat{\title}{\bfseries\Large}{\thesection}{2em}{}
\usepackage{titling}
\pretitle{\begin{center}\Large\bfseries}
\posttitle{\par\end{center}}
\title{Mapping Occupational Traits from the Third Edition of the Dictionary of Occupational Titles to the European ISCO-08 Classification\thanks{This document serves as an auxiliary source of information on the data used for the analysis in the paper \textit{``Working Conditions and Health at Work Over the Lifecycle: Evidence from Europe"}.}}

\author{Belloni, Lucifora, Micera} 

\date{}


\begin{document}
\setlength{\droptitle}{-5em}   % This will reduce the space before the title

\maketitle
\vspace{-6.2em}

\begin{center}  \href{https://github.com/paolomicera/DOT3rd-ISCO08/blob/main/main.pdf}{\textbf{Most Updated Version}}
\end{center}
\section{Introduction}
This document provides detailed information on the collection of data on working conditions from the digitalized version of the third edition of the Dictionary of Occupational Titles (DOT) and its conversion to the European classification system ISCO-08. 
The Dictionary of Occupational Titles (DOT) was first published in 1939 and underwent four major updates: in 1949, 1965, 1977, and its final revision in 1991. Since then, it has been replaced by the Occupational Information Network (O*NET). The DOT is a comprehensive classification system of occupational information, which includes a wide range of variables, such as job titles, job descriptions, and occupational characteristics. The DOT was developed by the U.S. Department of Labor and was used for various purposes, such as job placement, vocational guidance, and labor market analysis. The contribution of this document is twofold. First, it provides a detailed description of the data collection process from the digitalized version of the 1965 third edition of the DOT. Second, it describes the mapping of the occupational traits from the DOT to the European classification system ISCO-08.

\section{\textit{Dictionary of Occupational Titles} - Data Collection}
The data on occupational traits were collected from the digitalized version of the first supplement of the 1965 third edition of the DOT. This supplement contains more than 13,000 job titles, including job descriptions, job requirements, and occupational characteristics. The DOT is organized to provide a 9-digits code for each occupation where the first digits divides them into nine major categories, which are further divided into subcategories at three digits level. The nine major categories are: (1) Professional, Technical, and Managerial Occupations; (2) Clerical and Sales Occupations; (3) Service Occupations; (4) Agricultural, Fishery, and Forestry Occupations; (5) Processing Occupations; (6) Machine Trades Occupations; (7) Benchwork Occupations; (8) Structural Work Occupations; and (9) Miscellaneous Occupations. Each job title in the DOT is assigned a unique code, which is used to identify the job title in the DOT. While the first three digits of the DOT code are for the  occupational group the job belongs to. The digits from 4 to 6 contain information on how the job relates to data, people, and things, as they specifically indicate the job’s relationship to Data, People, and Things, respectively. These digits classify the job based on the highest level of complexity required for each of these interactions. The last three digits of the DOT code are used to identify the job title within the occupational group.

The DOT also contains detailed information on the job requirements and occupational characteristics of each job title.The job requirements include the physical demand, exposure to specific hazards or conditions, and psychosocial requirements, as well as the educational and training requirements. To match the demand of specific worker profile, D provides detailed information on the skills, capacities, and work preferences essential for effective job performance. It also includes data on the levels of specialized training time required to qualify for each occupation.













\newpage
\nocite{*}
\printbibliography

\end{document}
